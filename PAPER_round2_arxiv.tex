\documentclass[11pt,a4paper]{article}

% ─── Packages ───
\usepackage[utf8]{inputenc}
\usepackage[T1]{fontenc}
\usepackage{amsmath,amssymb,amsthm}
\usepackage{mathtools}
\usepackage{booktabs}
\usepackage{array}
\usepackage{hyperref}
\usepackage[margin=2.5cm]{geometry}
\usepackage{enumitem}
\usepackage{authblk}

% ─── Theorem environments ───
\newtheorem{theorem}{Theorem}[section]
\newtheorem{lemma}[theorem]{Lemma}
\newtheorem{proposition}[theorem]{Proposition}
\newtheorem{corollary}[theorem]{Corollary}
\newtheorem{conjecture}[theorem]{Conjecture}
\theoremstyle{definition}
\newtheorem{definition}[theorem]{Definition}
\theoremstyle{remark}
\newtheorem{remark}[theorem]{Remark}

% ─── Title ───
\title{On the Non-Existence of $[22,6,13]$ Codes over GF(4):\\
A Computational Investigation via Distributed AI Collaboration}

\author[1]{Rafael Amichis Luengo}
\author[2]{Claude (Anthropic)}
\author[2]{Gemini (Google)}
\author[2]{ChatGPT (OpenAI)}
\author[2]{Grok (xAI)}

\affil[1]{Proyecto Estrella, Madrid, Spain}
\affil[2]{Artificial Intelligence Systems (collaborative research agents)}

\date{February 21, 2026}

% ═══════════════════════════════════════
\begin{document}
\maketitle

% ─── Abstract ───
\begin{abstract}
We present the results of the most comprehensive computational investigation to date of the open problem $d_4(22,6) \in \{12, 13\}$---determining whether a $[22,6,13]$ linear code over $\mathrm{GF}(4)$ exists. This gap in the coding theory tables has remained open since at least 2001. Over a 48-hour campaign using four AI systems coordinated by a human director, we systematically eliminated $14$+ construction routes through approximately $60$~million code evaluations, proved five impossibility theorems, and introduced a novel extension attack methodology (``La P\'ua del Jet''). Our best construction achieves parameters $[22,6,12]_4$ with exactly 78 minimum-weight codewords ($A_{12} = 78$), exhibiting a perfect three-fold symmetry. We prove the \textbf{Collision Symmetry Theorem}: for any $[22,5,d_0 \geq 13]_4$ base code extended by a sixth row, the number of weight-12 codewords is always divisible by~3. Based on our evidence, we conjecture that $d_4(22,6) = 12$.
\end{abstract}

\noindent\textbf{Keywords:} linear codes, finite fields, GF(4), optimal codes, Griesmer bound, quasi-twisted codes, extension attacks, distributed AI research, error-correcting codes

\noindent\textbf{MSC2020:} 94B05, 94B65, 51E22

% ═══════════════════════════════════════
\section{Introduction}

\subsection{The Problem}

Let $d_q(n,k)$ denote the maximum minimum distance of a linear $[n,k]$ code over $\mathrm{GF}(q)$. The function $d_4(22,6)$ has been listed with bounds $12 \leq d_4(22,6) \leq 13$ in the online tables of Grassl~\cite{grassl} since at least December 2001, making it one of the longest-standing open entries for quaternary codes of moderate length.

The lower bound $d_4(22,6) \geq 12$ is achieved by shortening the quadratic residue code $\mathrm{QR}(29)$ over $\mathrm{GF}(4)$~\cite{macwilliams}. The upper bound $d_4(22,6) \leq 13$ follows from the Griesmer bound~\cite{griesmer}, which gives $n_{\min}(6,13,4) = 21$, so a $[22,6,13]_4$ code would have Griesmer slack~1.

\subsection{Contributions}

This paper presents:
\begin{enumerate}[nosep]
\item A systematic elimination map of $14$+ independent construction routes, each verified by exhaustive search or algebraic proof (Section~\ref{sec:elim}).
\item Five impossibility theorems closing specific construction approaches (Section~\ref{sec:impossibility}).
\item The \emph{Collision Symmetry Theorem}---a structural result on weight-12 codewords in extension codes over $\mathrm{GF}(4)$ (Section~\ref{sec:symmetry}).
\item The \emph{La P\'ua del Jet} methodology---a novel extension attack framework for codes with Griesmer slack (Section~\ref{sec:pua}).
\item The best known $[22,6,12]_4$ code with $A_{12} = 78$ and complete diagnostic data (Section~\ref{sec:matrix}).
\item A conjecture that $d_4(22,6) = 12$, supported by exhaustive computational evidence (Section~\ref{sec:conjecture}).
\end{enumerate}

% ═══════════════════════════════════════
\section{Preliminaries}

\subsection{Notation}

We denote $\mathrm{GF}(4) = \{0, 1, \omega, \omega^2\}$ where $\omega^2 + \omega + 1 = 0$. An $[n,k,d]_4$ code is a $k$-dimensional subspace of $\mathrm{GF}(4)^n$ with minimum Hamming distance~$d$. We write $A_w$ for the number of codewords of weight~$w$, and $\mathrm{wt}(c)$ for the Hamming weight of $c$.

\subsection{Griesmer Bound}

For a linear $[n,k,d]_q$ code:
\begin{equation}
n \;\geq\; g_q(k,d) \;=\; \sum_{i=0}^{k-1} \left\lceil \frac{d}{q^i} \right\rceil
\end{equation}
For our parameters: $g_4(6,13) = 13 + 4 + 1 + 1 + 1 + 1 = 21$. Since $22 > 21$, the Griesmer bound does not prohibit $[22,6,13]_4$.

\subsection{LP Feasibility}

The Delsarte linear programming bound~\cite{delsarte} does not exclude $[22,6,13]_4$. MacWilliams identity analysis confirms LP-feasibility with constraint $B_1 = B_2 = B_3 = B_4 = 0$ on the dual weight enumerator, yielding a dual code $[22,16,\geq 5]_4$.

If $[22,6,13]_4$ exists, $3 \leq A_{13} \leq 1{,}242$.

\subsection{Factorization of $x^{11} + 1$ over GF(4)}

\begin{equation}
x^{11} + 1 = (x+1) \cdot f_1(x) \cdot f_2(x)
\end{equation}
where $f_1(x) = 1 + \omega^2 x + x^2 + x^3 + \omega x^4 + x^5$ and $f_2(x) = 1 + \omega x + x^2 + x^3 + \omega^2 x^4 + x^5$ are conjugate irreducibles of degree~5, related by the Frobenius automorphism $\alpha \mapsto \alpha^2$.

% ═══════════════════════════════════════
\section{Elimination Map}\label{sec:elim}

We systematically tested $14$+ independent construction routes. Each route was verified by exhaustive enumeration or algebraic proof.

\subsection{Phase 2 Routes (12 routes, ${\sim}2$M evaluations)}

\begin{table}[h]
\centering
\small
\begin{tabular}{clcr}
\toprule
\# & Route & Result & Evaluations \\
\midrule
1 & QR(29) subcode: all 5{,}461 hyperplanes & $d=12$ & $1{,}100{,}000$ \\
2 & Puncture $[23,6,13]$: all 23 positions & $d=12$ & 23 \\
3 & Puncture $[24,7,13]$: all 24 positions & $d=12$ & 24 \\
4 & Construction X from $[21,5,13]$ & $d \leq 11$ & $8{,}000$+ \\
5 & Row extension from $[22,5,14]_4$ & $d=12$ & $500{,}000$+ \\
6 & Condensation $[23 \to 22]$ & $d=12$ & 759 \\
7 & Dual search: random $H$ & $d=12$ & $10{,}150$ \\
8 & Trace $\mathrm{GF}(16) \to \mathrm{GF}(4)$ & $d=12$ & $16{,}365$ \\
9 & Additive $\mathrm{GF}(2)$-linear & $d=12$ & $50{,}500$ \\
10 & Hill-climbing / Random / QC & $d=12$ & $703{,}000$+ \\
11 & CSP column-by-column & collapse at 14/16 & ${\sim}800$ \\
12 & QT same factor $f_1(x)$ & $d \leq 12$ (proved) & N/A \\
\bottomrule
\end{tabular}
\caption{Phase 2 elimination routes.}
\end{table}

\subsection{Round Two Routes (new in this work)}

\begin{table}[h]
\centering
\small
\begin{tabular}{clcr}
\toprule
\# & Route & Result & Evaluations \\
\midrule
13 & QT hybrid $f_1/f_2$, all 11 shifts & $d=12$ & $46{,}126{,}047$ \\
14 & Constacyclic $\lambda = \omega, \omega^2$ & $d=12$ & $8{,}380{,}000$ \\
15 & La P\'ua del Jet (extension attack) & $d=12$, $A_{12}{=}78$ & $3{,}248{,}026$ \\
\bottomrule
\end{tabular}
\caption{Round Two elimination routes.}
\end{table}

\textbf{Combined total: ${\sim}60$ million code evaluations.}

\subsubsection{Route 13: QT Hybrid with Conjugate Factors}

This route was identified as the highest-priority unexplored avenue because the algebraic proof for Route~12 showed that the $d \leq 12$ ceiling \emph{does not apply} when conjugate factors $f_1(x)$ and $f_2(x)$ are used. We exhaustively searched all QT codes of the form
\[
G = [\mathrm{circ}(a(x) \cdot f_1(x)),\; \mathrm{circ}(b(x) \cdot f_2(x) \cdot x^s)]
\]
across all 11 twist shifts $s \in \{0,\ldots,10\}$ and approximately 46~million rank-6 codes.

\textbf{Result:} Maximum achieved $d = 12$.

% ═══════════════════════════════════════
\section{Impossibility Theorems}\label{sec:impossibility}

\begin{theorem}\label{thm:hyperplane}
No hyperplane subcode of the $[22,7,12]_4$ code derived from $\mathrm{QR}(29)$ achieves minimum distance $\geq 13$.
\end{theorem}

\begin{proof}
The 921 codewords of weight~12 in $[22,7,12]_4$ span the full 7-dimensional space. Therefore every dimension-6 subcode contains at least one weight-12 codeword. Verified exhaustively over all 5{,}461 hyperplane subcodes.
\end{proof}

\begin{theorem}\label{thm:punct23}
The code $[23,6,13]_4$ cannot be punctured at any coordinate to yield a code with $d \geq 13$.
\end{theorem}

\begin{proof}
The 174 codewords of weight~13 have supports covering all 23 coordinates. Puncturing at any position reduces some weight-13 word to weight~12.
\end{proof}

\begin{theorem}\label{thm:punct24}
The code $[24,7,13]_4$ cannot be punctured at any single coordinate to yield $d \geq 13$.
\end{theorem}

\begin{proof}
Analogous to Theorem~\ref{thm:punct23}; the 384 weight-13 words cover all 24 coordinates.
\end{proof}

\begin{theorem}[ChatGPT]\label{thm:qt}
For a quasi-twisted code of index~2 over $\mathrm{GF}(4)$ with block length~11, if both constituent polynomials are multiples of the same irreducible factor $f_1(x)$ of $x^{11} + 1$, then $d \leq 12$.
\end{theorem}

\begin{proof}[Proof sketch]
The QT structure forces both circulant blocks into the ideal generated by $f_1(x)$ in $\mathrm{GF}(4)[x]/(x^{11}+1)$. The resulting code is contained in $C_1 \oplus C_2$ where each $C_i$ is a cyclic code with $d(C_i) \leq 6$. By the direct sum distance bound, $d \leq 2 \times 6 = 12$.
\end{proof}

\begin{theorem}\label{thm:griesmer}
The code $[22,7,13]_4$ does not exist.
\end{theorem}

\begin{proof}
The one-step Griesmer bound: $g_4(7,13) = 22$ with equality, and the divisibility condition is not met.
\end{proof}

% ═══════════════════════════════════════
\section{The Collision Symmetry Theorem}\label{sec:symmetry}

\begin{theorem}[Collision Symmetry]\label{thm:collision}
Let $C_0$ be a $[22,5,d_0]_4$ code with $d_0 \geq 13$, and let $g_6 \in \mathrm{GF}(4)^{22} \setminus C_0$ be any vector such that $C = \langle C_0, g_6 \rangle$ has dimension~6. Define
\[
A_{12}(\alpha) \;=\; \bigl|\{c_0 \in C_0 : \mathrm{wt}(c_0 + \alpha \cdot g_6) = 12\}\bigr|
\]
for $\alpha \in \mathrm{GF}(4)^*$. Then $A_{12}(1) = A_{12}(\omega) = A_{12}(\omega^2)$, and consequently the total number of weight-12 codewords in $C$ is divisible by~3.
\end{theorem}

\begin{proof}
The map $\varphi_\alpha\!: c_0 \mapsto c_0$ on $C_0$ is the identity, and scalar multiplication by $\omega$ is an automorphism of $(\mathrm{GF}(4), +)$. For each $c_0 \in C_0$ and each coordinate $j$ where $g_6[j] \neq 0$:
\[
c_0[j] + \alpha \cdot g_6[j] = 0 \quad\Longleftrightarrow\quad c_0[j] = \alpha \cdot g_6[j]
\]
Since the map $\alpha \mapsto \omega\alpha$ permutes $\mathrm{GF}(4)^*$ cyclically, the weight distributions over the three cosets $\{c_0 + \alpha g_6 : c_0 \in C_0\}$ for $\alpha = 1, \omega, \omega^2$ are identical. Specifically, the number of coordinates where $c_0[j] + \alpha g_6[j] = 0$ has the same distribution for each $\alpha \in \mathrm{GF}(4)^*$.
\end{proof}

\begin{corollary}
The collision count $A_{12}$ is always $\equiv 0 \pmod{3}$. To achieve $d = 13$, one requires $A_{12} = 0$ exactly---a discrete jump from the observed minimum of 78.
\end{corollary}

\subsection{Empirical Verification}

In V-FINAL-1, across 1{,}548{,}008 complete evaluations with simulated annealing, the optimal sixth row produced $A_{12}(1) = A_{12}(\omega) = A_{12}(\omega^2) = 26$, giving total $A_{12} = 78$. This symmetry was observed consistently across all candidate rows.

% ═══════════════════════════════════════
\section{The La P\'ua del Jet Methodology}\label{sec:pua}

\subsection{Concept}

We introduce an extension attack framework for codes with Griesmer slack $\geq 1$, proceeding in three phases:

\textbf{Phase 1 --- Fuselage Construction.} Construct a $[n, k{-}1, d']$ code $C_0$ with $d' \geq d_{\mathrm{target}}$.

\textbf{Phase 2 --- Fuselage Freeze.} Fix $C_0$ permanently.

\textbf{Phase 3 --- Spike Injection.} Search for $g_k \in \mathrm{GF}(q)^n \setminus C_0$ such that $C = \langle C_0, g_k \rangle$ has $d(C) \geq d_{\mathrm{target}}$, using simulated annealing with multi-position mutations.

\subsection{Rationale}

The name derives from an aerospace engineering metaphor: an aerospike on a hypersonic vehicle creates a forward shockwave to reduce drag. The $k$-th row actively ``clears'' low-weight codewords from the linear combinations involving it.

\subsection{Implementation}

For $[22,6,13]_4$: base $[22,5,13]_4$ constructed by hill-climbing (2{,}401 attempts, 57~seconds); all $4^5 = 1{,}024$ base codewords precomputed; for each candidate $g_6$, all $3 \times 1{,}024 = 3{,}072$ new codewords verified; simulated annealing with single/multi-position mutations and 8~restarts.

\begin{table}[h]
\centering
\begin{tabular}{lcrr}
\toprule
Engine & Strategy & Evaluations & $A_{12}$ \\
\midrule
V-FINAL-1 & Single base, SA, 8 restarts & 1{,}548{,}008 & \textbf{78} \\
V-FINAL-2 & 5 bases, SA per base & 1{,}700{,}018 & 93 \\
\bottomrule
\end{tabular}
\caption{La P\'ua engine results.}
\end{table}

% ═══════════════════════════════════════
\section{Best Known Code}\label{sec:matrix}

\subsection{Generator Matrix}

The generator of the best $[22,6,12]_4$ found, encoding $0{=}0$, $1{=}1$, $2{=}\omega$, $3{=}\omega^2$:

\[
G = \begin{pmatrix}
1 & 0 & 0 & 0 & 0 & 2 & 3 & 2 & 2 & 3 & 3 & 2 & 1 & 0 & 1 & 1 & 3 & 0 & 2 & 0 & 0 & 1 \\
0 & 1 & 0 & 0 & 0 & 1 & 2 & 0 & 2 & 1 & 2 & 3 & 0 & 3 & 1 & 2 & 3 & 1 & 0 & 1 & 0 & 2 \\
0 & 0 & 1 & 0 & 0 & 3 & 3 & 2 & 2 & 1 & 2 & 3 & 2 & 1 & 3 & 0 & 1 & 3 & 1 & 2 & 1 & 2 \\
0 & 0 & 0 & 1 & 0 & 0 & 2 & 0 & 3 & 3 & 3 & 0 & 1 & 2 & 2 & 1 & 0 & 0 & 3 & 2 & 2 & 1 \\
0 & 0 & 0 & 0 & 1 & 1 & 3 & 3 & 0 & 1 & 0 & 3 & 1 & 1 & 2 & 2 & 3 & 0 & 0 & 3 & 2 & 1 \\
1 & 0 & 2 & 0 & 2 & 1 & 2 & 3 & 0 & 0 & 3 & 3 & 0 & 3 & 0 & 2 & 1 & 2 & 3 & 0 & 2 & 2
\end{pmatrix}
\]

Rows 1--5 generate a $[22,5,13]_4$ subcode. Row~6 is the ``P\'ua'' vector. All 78 weight-12 codewords involve row~6 with nonzero coefficient.

\subsection{Weight Distribution}

\begin{table}[h]
\centering
\begin{tabular}{cc|cc|cc}
\toprule
$w$ & $A_w$ & $w$ & $A_w$ & $w$ & $A_w$ \\
\midrule
12 & \textbf{78} & 16 & 672 & 20 & 213 \\
13 & 279 & 17 & 771 & 21 & 42 \\
14 & 348 & 18 & 669 & 22 & 3 \\
15 & 585 & 19 & 435 & & \\
\bottomrule
\end{tabular}
\caption{Weight distribution of $[22,6,12]_4$ (4{,}095 nonzero codewords).}
\end{table}

\subsection{Coordinate Vulnerability}

For each coordinate $j$, the number $z(j)$ of weight-12 words with zero at position~$j$:
\[
\begin{array}{l}
z = (27, 36, 36, 36, 39, 36, 30, 36, 42, 27, 27, 27, 36, 42, 51, 39, 42, 33, 33, 33, 33, 39)
\end{array}
\]
Minimum: $z = 27$ (positions 0, 9, 10, 11). Maximum: $z(14) = 51$. Since all $z(j) > 0$, no single-coordinate puncture yields $d \geq 13$.

% ═══════════════════════════════════════
\section{Conjecture}\label{sec:conjecture}

\begin{conjecture}
$d_4(22,6) = 12$. The linear code $[22,6,13]_4$ does not exist.
\end{conjecture}

This is supported by: (1)~exhaustive elimination of 14+ construction routes; (2)~${\sim}60$~million stochastic evaluations; (3)~five impossibility theorems; (4)~the Collision Symmetry Theorem; (5)~25 years of community research without a construction.

A definitive resolution could come from SAT/ILP formulation (96~variables, 4{,}095 constraints), semidefinite programming bounds, or matroid-theoretic arguments.

% ═══════════════════════════════════════
\section{Methodological Notes}

This research was organized as a hub-and-spoke collaboration with the human researcher as hub and four AI systems as spokes. The human coordinator---a psychology graduate with no formal mathematics training---contributed creative metaphors that revealed structural features (``babies and bodyguards'' for minimum distance, ``La P\'ua del Jet'' for extension attacks), persistence beyond AI convergence, and quality control identifying errors across all four systems.

% ─── References ───
\begin{thebibliography}{10}

\bibitem{grassl}
M.~Grassl.
\newblock Bounds on the minimum distance of linear codes and quantum codes.
\newblock Online: \url{https://codetables.de}. Accessed February 2026.

\bibitem{macwilliams}
F.~J.~MacWilliams and N.~J.~A.~Sloane.
\newblock {\em The Theory of Error-Correcting Codes}.
\newblock North-Holland, 1977.

\bibitem{griesmer}
J.~H.~Griesmer.
\newblock A bound for error-correcting codes.
\newblock {\em IBM J. Res. Dev.}, 4(5):532--542, 1960.

\bibitem{gulliver}
T.~A.~Gulliver and V.~K.~Bhargava.
\newblock Some best rate $1/p$ and rate $(p{-}1)/p$ systematic quasi-cyclic codes over GF(3) and GF(4).
\newblock {\em IEEE Trans. Inf. Theory}, 38(4):1369--1374, 1992.

\bibitem{delsarte}
P.~Delsarte.
\newblock An algebraic approach to the association schemes of coding theory.
\newblock {\em Philips Res. Rep. Suppl.}, 10, 1973.

\bibitem{brouwer}
A.~Brouwer.
\newblock Bounds on linear codes.
\newblock In {\em Handbook of Coding Theory}, V.~S.~Pless and W.~C.~Huffman (eds.), vol.~1, ch.~4. Elsevier, 1998.

\end{thebibliography}

\end{document}
